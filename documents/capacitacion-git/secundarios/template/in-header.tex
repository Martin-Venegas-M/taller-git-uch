%-----------------------
%	Cargar paquetes y otras modificaciones
%-----------------------

% Paquetes
\usepackage{fancyhdr}
\usepackage{ifthen}
\usepackage{xcolor} % Cargar paquete xcolor con nombres de colores
\usepackage{etoolbox} % Modificar el comportamiento de \chapter
\usepackage{sectsty} % Editar títulos
\usepackage{caption} % Configuración de leyendas
\usepackage{array} % Definir nuevos tipos de columnas en tablas
\usepackage{titlesec} % Modificar títulos
\usepackage{sidenotes}
\usepackage{xparse}
\usepackage{fontspec} % Para cargar cada fuente
\usepackage{tocloft}  % Paquete para personalizar ToC, LoF y LoT
% \usepackage{hyperref} % Para cambiar formato de referencias
\usepackage{tikz} % Para crear gráficos y dibujos
\usepackage{geometry} % Para controlar dimensiones de la página y márgenes
\usepackage{graphicx} % Para manipular gráficos

% Colores
\definecolor{colorCUT}{HTML}{8760ac}
\definecolor{colorCH}{HTML}{2d93ad}

\usepackage[table]{xcolor}
\definecolor{lightpurple}{HTML}{DECBE4}

% Fuente
\setmainfont{Bahnschrift.ttf}

%-----------------------
%	CONFIGURAR ToC, LoF, LoT
%-----------------------

% TOC
% Configura el fondo del título de la tabla de contenidos
\renewcommand{\cfttoctitlefont}{%
    \color{white} % Cambia el color del texto del título
    \colorbox{colorCUT}{%
        \makebox[\textwidth][l]{% Ajusta el tamaño del colorbox
            \Large\bfseries\textcolor{white}{\contentsname}%
        }%
    }%
}

% Ajusta el espaciado de la tabla de contenidos
\renewcommand{\cftaftertoctitleskip}{0pt} % Elimina el espacio después del título de la tabla de contenidos

% Asegúrate de que no hay texto adicional
\renewcommand{\cftaftertoctitle}{\vspace{-\baselineskip}\hfill} % Añade un \hfill para asegurar que no haya texto extra

% LOF
% Configura el fondo del título de la lista de figuras
\renewcommand{\cftloftitlefont}{%
    \color{white} % Cambia el color del texto del título
    \colorbox{colorCUT}{%
        \makebox[\textwidth][l]{% Ajusta el tamaño del colorbox
            \Large\bfseries\textcolor{white}{\listfigurename}%
        }%
    }%
}

% Ajusta el espaciado de la lista de figuras
\renewcommand{\cftafterloftitleskip}{0pt} % Elimina el espacio después del título de la lista de figuras

% LOT
% Configura el fondo del título de la lista de tablas
\renewcommand{\cftlottitlefont}{%
    \color{white} % Cambia el color del texto del título
    \colorbox{colorCUT}{%
        \makebox[\textwidth][l]{% Ajusta el tamaño del colorbox
            \Large\bfseries\textcolor{white}{\listtablename}%
        }%
    }%
}

% Ajusta el espaciado de la lista de tablas
\renewcommand{\cftafterlottitleskip}{0pt} % Elimina el espacio después del título de la lista de tablas

% Asegúrate de que no haya texto adicional
\renewcommand{\cftafterlottitle}{\vspace{-\baselineskip}} % Ajusta el espacio después del título

% Personaliza ToC, LoF y LoT
\makeatletter
% Eliminar el salto de página en \listoftables
\patchcmd{\listoftables}{\clearpage}{}{}{} % Remueve el salto de página después de la lista de tablas

% Evitar la creación de páginas en blanco después de ToC y LoF
\let\oldtableofcontents\tableofcontents
\renewcommand{\tableofcontents}{
  \oldtableofcontents
  \afterpage{\clearpage} % Asegura que no haya página en blanco después de ToC
}

\let\oldlistoffigures\listoffigures
\renewcommand{\listoffigures}{
  \oldlistoffigures
  \afterpage{\clearpage} % Asegura que no haya página en blanco después de LoF
}
\makeatother

%-----------------------
%	CONFIGURAR PIE DE PÁGINA
%-----------------------

% Estilo general para el pie de página en todo el documento
\pagestyle{fancy}
\fancyhf{}
\fancyfoot[C]{\raisebox{-\footrulewidth}{\colorbox{colorCH}{\textbf{\textcolor{white}{\thepage}}}}}
\renewcommand{\footrulewidth}{1.25pt}

% Definir las reglas del pie de página para páginas pares e impares
\newcommand{\evenfootrule}{%
  {
   \color{white}\rule{1.2\linewidth}{\footrulewidth}%
   }}

\newcommand{\oddfootrule}{%
  {
   \color{white}\rule{1.2\linewidth}{\footrulewidth}%
   }}

\renewcommand{\footrule}{%
  \ifodd\value{page}%
    \oddfootrule%
  \else%
    \evenfootrule%
  \fi%
}

% Estilo especial para las primeras páginas de los capítulos
\fancypagestyle{chapterfirst}{
  \fancyhf{} % Limpiar encabezado y pie de página
  \fancyfoot[C]{\raisebox{-\footrulewidth}{\colorbox{colorCH}{\textbf{\textcolor{white}{\thepage}}}}}
  \renewcommand{\footrulewidth}{1.25pt}
}

% Aplica el estilo a la primera página del primer capítulo (página inicial del documento)
\fancypagestyle{plain}{
  \fancyhf{} % Limpia encabezados y pies de página
  \renewcommand{\footrulewidth}{0pt} % Elimina la línea del pie de página
}


%-----------------------
%	HEADER
%-----------------------

% Definir un nuevo comando para almacenar el texto predefinido
% Modificar para cada documento
\newcommand{\chapterTitle}{Documento de Principales Resultados - II ENUT}

% Limpiar encabezados y pies de página
\fancyhead{}

% Definir encabezado para páginas impares y pares
\fancyhead[LO,RE]{\textcolor{colorCH}{\chapterTitle}}
\fancyhead[C]{}
\fancyhead[RO,LE]{}

% Ancho de la línea del encabezado
\renewcommand{\headrulewidth}{0.75pt}

% Definir la línea horizontal en páginas pares e impares
\newcommand{\evenheadrule}{%
  {
  \color{white}\rule{0.3\headwidth}{1.25pt}%
  \color{colorCH}\rule{0.7\headwidth}{1.25pt}%
}}

\newcommand{\oddheadrule}{%
  {
  \color{colorCH}\rule{0.7\headwidth}{1.25pt}%
  \color{white}\rule{0.3\headwidth}{1.25pt}%
}}

% Redefinir la regla del encabezado según si la página es par o impar
\renewcommand{\headrule}{%
  \ifodd\value{page}%
    \oddheadrule%
  \else%
    \evenheadrule%
  \fi%
}

% Estilo especial para las primeras páginas de los capítulos
\fancypagestyle{chapterfirst}{
  \fancyhf{} % Limpiar encabezado y pie de página
  \fancyhead[LO,RE]{\textcolor{colorCH}{\chapterTitle}} % Encabezado
  \fancyhead[C]{}
  \fancyhead[RO,LE]{}
  \fancyfoot[C]{\raisebox{-\footrulewidth}{\colorbox{colorCH}{\textbf{\textcolor{white}{\thepage}}}}}
  \renewcommand{\headrulewidth}{0.75pt}
  \renewcommand{\footrulewidth}{1.25pt}
  \renewcommand{\headrule}{%
    \ifodd\value{page}%
      \oddheadrule%
    \else%
      \evenheadrule%
    \fi%
  }
}

% Aplicar estilo solo en las primeras páginas de los capítulos
%%% Funciona el header y el footer %%%
\patchcmd{\chapter}{plain}{chapterfirst}{}{}


%-----------------------
%	TÍTULOS
%-----------------------

% Redefinir la numeración de capítulos con ceros a la izquierda
% \makeatletter
% \renewcommand{\thechapter}{\ifnum\value{chapter}<10 0\fi\arabic{chapter}}
% \makeatother

% Redefinir el formato de capítulos
\makeatletter
\renewcommand{\@makechapterhead}[1]{%
  {%
    \noindent % Asegurarse de que no haya sangría en la línea
    \colorbox{colorCUT}{\textcolor{white}{\normalfont\Huge \csname thechapter\endcsname}}%
    \hspace{1em} % Espacio horizontal entre el colorbox y el título del capítulo
    {\normalfont\Huge\bfseries #1} % Formato del título
    \par\nobreak
    \vspace{2em} % Espacio después del título
  }%
}
\makeatother


%-----------------------
%	SUB-TÍTULOS
%-----------------------

% Redefinir el formato de subsecciones
\makeatletter
\renewcommand{\@seccntformat}[1]{%
  \ifnum\value{#1}=0\else%
    \textcolor{colorCUT}{\csname the#1\endcsname.\hspace{0.5em}}%
  \fi
}
\makeatother

% Personalizar el formato del título de subsecciones
\titleformat{\subsection}[runin]
  {\mdseries\normalsize \color{black}} % Formato del título (texto en negro)
  {\textcolor{colorCUT}{\thesubsection}} % Formato del número (en colorCUT)
  {0.5em} % Espacio entre el número y el título
  {} % No se agrega texto adicional antes del título

%-----------------------
%	TABLES
%-----------------------

\DeclareCaptionLabelSeparator{narrow}{\hspace{0.5pt}} % Define un espacio muy pequeño

\captionsetup[table]{
    skip=1pt,
    font={bf,normalsize},
    labelsep=narrow,
    labelfont={color=white},
    textfont={color=colorCUT},
    format=plain,
    justification=centering,
    singlelinecheck=false,
    name={\colorbox{colorCUT}{\textcolor{white}{\textbf{Tabla \thetable}}}}
}

\renewcommand{\thetable}{\arabic{table}} % Definir el formato del contador de tablas

%-----------------------
%	FIGURAS
%-----------------------

\captionsetup[figure]{
    skip=1pt,
    font={bf,normalsize},
    labelsep=narrow,
    labelfont={color=white},
    textfont={color=colorCUT},
    format=plain,
    justification=centering,
    singlelinecheck=false,
    name={\colorbox{colorCUT}{\textcolor{white}{\textbf{Figura \thefigure}}}}
}

\renewcommand{\thefigure}{\arabic{figure}} % Formato del contador de figuras

%-----------------------
%	NOTAS AL LADO
%-----------------------

% Ejemplo corregido para notas al lado (sidenotes)
\makeatletter
\RenewDocumentCommand\sidenotetext{o o +m}{%
    \IfNoValueOrEmptyTF{#1}{%
        \@sidenotes@placemarginal{#2}{\textsuperscript{\thesidenote}\footnotesize #3}%
        \refstepcounter{sidenote}%
    }{%
        \@sidenotes@placemarginal{#2}{\textsuperscript{#1}\footnotesize #3}%
    }%
}
\makeatother

% Opciones adicionales para notas sin numeración
\newcommand{\nonumsidenote}[2][]{\sidenote[\hspace{30pt}][#1]{\footnotesize #2}}
\newcommand{\sidenotequote}[2][]{\sidenote[\hspace{30pt}][#1]{\rmfamily\itshape #2}}

%-----------------------
%	NOTAS AL PIE
%-----------------------

\makeatletter
% Cambiar el color y poner en negrita el número de la nota al pie
\renewcommand\@makefnmark{%
    \textsuperscript{\textbf{\textcolor{colorCH}{\@thefnmark}}}% Número en negrita y en colorCH
}

% Cambiar el color del texto de la nota al pie
\renewcommand\@makefntext[1]{%
  \noindent\makebox[0pt][r]{\@makefnmark\ }%
  \textcolor{colorCH}{#1}% Cambia el color del texto de la nota
}

% Cambiar el color de la línea de las notas al pie
\renewcommand\footnoterule{%
    \kern-1pt % Ajuste de la separación de la línea con el texto
    \color{colorCH}% Cambia el color de la línea
    \hrule width 0.4\columnwidth height 0.5pt % Ajusta el ancho y el grosor de la línea
    \kern 1pt % Ajuste del espacio debajo de la línea
}
\makeatother

%-----------------------
%	NOTAS AL PIE
%-----------------------
% Fijar colores para referencias
% 
% \usepackage{hyperref} % Para cambiar formato de referencias
% 
% \hypersetup{
%     colorlinks=true, % Activar colores
%     linkcolor=red,   % Color para enlaces internos
%     urlcolor=red,    % Color para URLs
%     citecolor=red    % Color para citas
% }

\usepackage{etoolbox}
\makeatletter
\patchcmd{\listoftables}{\clearpage}{}{}{} % Evita la página en blanco después de la lista de tablas
\makeatother

\usepackage{etoolbox}
\makeatletter
\patchcmd{\listoffigures}{\clearpage}{}{}{} % Evita la página en blanco después de la lista de figuras
\patchcmd{\tableofcontents}{\clearpage}{}{}{} % Evita la página en blanco después de la tabla de contenido
\patchcmd{\listoftables}{\clearpage}{}{}{} % Evita la página en blanco después de la lista de tablas
\makeatother
